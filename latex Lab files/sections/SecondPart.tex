\section{Rahul Soni}

\subsection{Comments about the module}
The lectures used a multitude of real world examples to link what we were doing to industry standards, which was great because a lot of it doesn't sound very susbstantial at first. Slides are laid out well and are easy to revise from when used in conjuction with notes. Lab sessions were tight for time, but I feel like that was at least a little intentional. The intro to group felt a little pointless as the chances I end up liking everyone in my group next year as much as I like the people I chose myself is next to none, and I feel like that will be the greatest hurdle next year. Something like a system where you pair up and get put with another pair would have struck a good middle ground. 

\subsection{Selfie with Max}

To include an image, you will need to remove the comments from the code below, place an image in the main folder, and do not forget to put the name of the image instead of ImgName. 

%\begin{figure}[h]
%\caption{Selfie with Max}
%\centering
%\includegraphics[width=0.5\textwidth]{ImgName}
%\label{fig:selfie}
%\end{figure}

You can then use the label of the figure to reference it later with the command ${\backslash}ref.$ you can comment out the next line to see an example of how it works.

% My selfie with Max is in  Figure~\ref{fig:selfie}.

\subsection{What I have learned in this module}
The main thing I am taking away from this module is that many people are good at programming, but precious few are good at designing and planning - which is where the real gains are made. The importance of things like proper specifications were underscored in this module a lot more than they were at A-level. In addition, the section on testing was very useful- up to this point testing for me consisted of running the code with print messages embedded to try and track the loops - this was a welcome change.

